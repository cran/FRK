\section{Introduction}\label{sec:intro}

Fixed Rank Kriging (FRK) is a spatial/spatio-temporal modelling and prediction framework that is scaleable, works well with large datasets, and can change spatial support easily. FRK hinges on the use of a spatial random effects (SRE) model, in which a spatially correlated mean-zero random process is decomposed using a linear combination of spatial basis functions with random weights plus a term that captures the random process' fine-scale variation. Dimensionality reduction through a relatively small number of basis functions ensures computationally efficient prediction, while the reconstructed spatial process is, in general, non-stationary.  The SRE model has a spatial covariance function that is always nonnegative-definite, and, because any (possibly non-orthogonal) basis functions can be used, it can be constructed so as to approximate standard families of covariance functions \citep{Kang_2011}.  For a detailed treatment of FRK, see \cite{Cressie_2006,Cressie_2008,Shi_2007}, and \cite{Nguyen_2012}.

There are numerous \proglang{R} packages available for modelling and prediction with spatial or spatio-temporal data,\footnote{see \url{https://cran.r-project.org/web/views/Spatial.html}.} although relatively few of these make use of a model with spatial basis functions. However, a few variants of FRK have been developed to date, and the one that comes closest to the present software is \pkg{LatticeKrig} \citep{Nychka_2015}. \pkg{LatticeKrig} uses Wendland basis functions (that have compact support) to decompose the spatially correlated process, and it also has a Markov assumption to construct a precision matrix  (the matrix $\Kmat^{-1}$ in Section \ref{sec:SREModel}) to describe the dependence between the coefficients of these basis functions. It does not cater for what we term fine-scale-process variation, and instead the finest scale of the process is limited to the finest resolution of the basis functions used. However, this scale can be relatively fine due to the computationally motivated sparsity imposed on $\Kmat^{-1}$.  \pkg{LatticeKrig}'s underlying model makes use of sparse precision matrices constructed using Gaussian Markov random field (GMRF) assumptions, which results in efficient computations and the potential use of a large number ($>10,000$) of basis functions. %Non-stationarity is limited in \pkg{LatticeKrig's} GMRFs to either a few parameters that control spatially-varying variance or to some pre-determined specification \citep[][p.~21]{LatticeKrig}.  However, heterogeneity is not straightforward in GMRFs, in general; for example inference on slowly varying length scales may be carried out using the approach of \citet{Lindgren_2011} but abrupt changes are difficult to characterise. % \red{There is one final feature of \pkg{LatticeKrig} that allows fast computation but results in an incoherent spatial model. The same random effect can be assigned to two spatial resolutions, where it is assumed statistically independent of itself \citep{Nychka_2015,Bradley_206}. This can only happen in the degenerate case where the random effect is in fact non-random.}

The package \pkg{INLA} is a general-purpose package for model fitting and prediction. When using \pkg{INLA} for spatial and spatio-temporal modelling, the prevalent approach is to assume that basis functions are triangular `tent' functions and that the coefficients are normally distributed with a sparse precision matrix, such that the covariance function of the resulting Gaussian process is approximately a spatial covariance function from the Mat{\'e}rn class \citep[see][for details on software implementation]{Lindgren_2015}. \pkg{INLA}'s approach thus shares many of the features of \pkg{LatticeKrig}. % In principle, standard FRK could also be implemented within \pkg{INLA}; however, this would require considerable extra implementation effort, especially if one is using spatio-temporal models.
A key advantage of \pkg{INLA} is that once the spatial or spatio-temporal model is constructed, one has access to all the approximate-inference machinery and likelihood models available within the package.

\cite{Kang_2011} develop Bayesian FRK; they keep the spatial basis functions fixed and put a prior distribution on $\Kmat$. The predictive-process approach of \citet{Banerjee_2008} can also be seen as a type of Bayesian FRK, where the basis functions are constructed from the postulated covariance function of the spatial random effects and hence depend on parameters \citep[see][for an equivalence argument]{Katzfuss_2014}. An \proglang{R} package that implements predictive processes is \pkg{spBayes} \citep{Finley_2007}. It allows for multivariate spatial or spatio-temporal processes, and Bayesian inference is carried out using Markov chain Monte Carlo (MCMC), thus allowing for a variety of likelihood models. Because the implied basis functions are constructed based on a parametric covariance model, a prior distribution on parameters reults in new basis functions generated at each MCMC iteration. Since this can slow down the computation, the number of knots used in predictive processes needs to be small.

Our software package \pkg{FRK} differs from spatial prediction packages currently available by constructing an SRE model on a discretised domain, where the discrete element is known as a basic areal unit \citep[BAU; see, e.g.,][]{Nguyen_2012}. Reverting to discretised spatial processes might appear to be counter-intuitive, given all the theory and efficient approaches available for continuous-domain processes. However, BAUs allow one to easily combine multiple observations with different supports, which is common when working with, for example, remote sensing datasets. We also find that when using sparse partial-matrix inversion approaches outlined in \citet{Takahashi_1973}, we are able to carry out exact predictions (without conditional simulation) at millions of prediction locations with relative ease. Further, the consideration of a discrete element allows one to distinguish between measurement error and fine-scale variation at the resolution of the discrete element which, as will be seen in this article, leads to better uncertainty quantification. The BAUs need to be `small,' in the sense that they should be able to reconstruct the (undiscretised) process with minimal error, but \pkg{FRK} implements functions to predict over any arbitrary user-defined polygons.

In the standard ``flavour'' of \pkg{FRK} \citep{Cressie_2008}, which we term \emph{vanilla} FRK (FRK-V), there is an explicit reliance on multi-resolution basis functions to give complex non-stationary spatial patterns at the cost of not imposing any structure on $\Kmat$, the covariance matrix of the basis function weights. This can result in identifiability issues and hence can result in over-fitting the data when $\Kmat$ is estimated using standard likelihood methods  \citep[e.g.,][]{Nguyen_2014}, especially in regions of data paucity. Therefore, in \pkg{FRK} we also implement a model (FRK-M) where a parametric structure is imposed on $\Kmat$ \citep[e.g.,][]{Stein_2008,Nychka_2015}. The main aim of the package \pkg{FRK} is to facilitate spatial and spatio-temporal analysis and prediction for large datasets, where multiple observatons come with different spatial supports. We see that in `big data' scenarios, lack of consideration of fine-scale variation may lead to over-confident predictions, irrespective of the number of basis functions adopted.

% based on what, from experience, we deem to be the most `usual' scenario faced by the analyst in the environmental sciences:
%
% \begin{itemize}
% \item The data has additive Gaussian measurement error, and is available at multiple spatial resolutions (or supports).
% \item The physical process being modelled is either spatial or spatio-temporal, on $\mathbb{R}^1, \mathbb{R}^2$ or $\mathbb{S}^2$, and can be heterogeneous in both space and time.
% \item There is a spatial (or spatio-temporal) unit, which we term a Basic Areal Unit (BAU), which is the smallest unit over which an observation is considered informative or the smallest unit over which we need to predict.
% \item Datasets can have hundreds of thousands of data points and hundreds of thousands of BAUs.
% \item The extent, or nature, of spatial heterogeneity can vary greatly between one application and the next. A useful class of models for the analyst would thus be  one which is quite basic but that can also offer considerably flexibility (such as the SRE class).
% \end{itemize}

